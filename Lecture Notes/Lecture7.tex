%%%%%%%%%%%%%%%%%%%%%%%%%%%%%%%%%%%%%%%%%%%%%%%%%%%%%%%%%%%%%%%%%%%%%%%%%%%%%%%%%%%%%%%%%%%%%%%%%%%%%%%%%%%%%%%%%%%%%%%%%%%%%%%%%%%%%%%%%%%%%%%%%%%%%%%%%%%%%%%%%%%%%%%%%%%%%%%%%%%%%%%%%%%%
% Written By Michael Brodskiy
% Class: Analytic Geometry & Calculus III (Math-292)
% Professor: V. Cherkassky
%%%%%%%%%%%%%%%%%%%%%%%%%%%%%%%%%%%%%%%%%%%%%%%%%%%%%%%%%%%%%%%%%%%%%%%%%%%%%%%%%%%%%%%%%%%%%%%%%%%%%%%%%%%%%%%%%%%%%%%%%%%%%%%%%%%%%%%%%%%%%%%%%%%%%%%%%%%%%%%%%%%%%%%%%%%%%%%%%%%%%%%%%%%%

\documentclass[12pt]{article} 
\usepackage{alphalph}
\usepackage[utf8]{inputenc}
\usepackage[russian,english]{babel}
\usepackage{titling}
\usepackage{amsmath}
\usepackage{graphicx}
\usepackage{enumitem}
\usepackage{amssymb}
\usepackage[super]{nth}
\usepackage{everysel}
\usepackage{ragged2e}
\usepackage{geometry}
\usepackage{fancyhdr}
\geometry{top=1.0in,bottom=1.0in,left=1.0in,right=1.0in}
\newcommand{\subtitle}[1]{%
  \posttitle{%
    \par\end{center}
    \begin{center}\large#1\end{center}
    \vskip0.5em}%

}
\usepackage{hyperref}
\hypersetup{
colorlinks=true,
linkcolor=blue,
filecolor=magenta,      
urlcolor=blue,
citecolor=blue,
}

\urlstyle{same}


\title{Lecture VII Notes}
\date{\today}
\author{Michael Brodskiy\\ \small Professor: V. Cherkassky}

% Mathematical Operations:

% Sum: $$\sum_{n=a}^{b} f(x) $$
% Integral: $$\int_{lower}^{upper} f(x) dx$$
% Limit: $$\lim_{x\to\infty} f(x)$$

\begin{document}

\maketitle

\subsection{Formulas From 13.3}

\begin{enumerate}

  \item $\overrightarrow{T}(t)=\frac{\overrightarrow{r}'(t)}{|\overrightarrow{r}'(t)|}$

  \item $\overrightarrow{N}(t)=\frac{\overrightarrow{T}'(t)}{|\overrightarrow{T}'(t)|}$

  \item $\overrightarrow{B}(t)=\overrightarrow{T}(t)\text{ x }\overrightarrow{N}(t)$

  \item $\kappa(t)=\frac{|\overrightarrow{T}'(t)|}{|\overrightarrow{r}'(t)|}=\frac{|\overrightarrow{r}'(t)\text{ x }\overrightarrow{r}''(t)|}{|\overrightarrow{r}'(t)|^3}$

\end{enumerate}

\section{Formulas of Frenet-Serret}

\begin{enumerate}

  \item $\frac{d\overrightarrow{T}}{ds}=\kappa\overrightarrow{N}$

  \item $\frac{d\overrightarrow{N}}{ds}=-\kappa\overrightarrow{T}+\tau\overrightarrow{B}$

  \item $\frac{d\overrightarrow{B}}{ds}=-\tau\overrightarrow{N}$  

\end{enumerate}

\section{Motion in Space $-$ 13.4}

\begin{enumerate}
 
  \item $\overrightarrow{v}(t)=\overrightarrow{r}'(t)$

  \item $\overrightarrow{a}(t)=\overrightarrow{v}'(t)=\overrightarrow{r}''(t)$

  \item $\text{speed}=|\overrightarrow{v}(t)|$

  \item The force, $\overrightarrow{F}$ is always acting in the same direction as the acceleration, $\overrightarrow{a}$, with proportionality constant mass, $m$

  \item In a circular path, an object's position is given by: $\overrightarrow{r}(t) = a\cos \omega t \hat{\bold{i}} + a \sin \omega t \hat{\bold{j}}$, where $a$ is the radius of the circle, and $\omega$ is the constant speed of the object. The velocity is given by: $\overrightarrow{r}'(t)=\overrightarrow{v}(t)=-a\omega \sin \omega t \hat{\bold{i}} + a \omega \cos \omega t \hat{\bold{j}}$, and the acceleration is given by: $\overrightarrow{a}(t)=\overrightarrow{v}'(t)=-a\omega^2\cos \omega t \hat{\bold{i}} - a\omega^2 \sin \omega t \hat{\bold{j}}$

  \item Also in circular motion, $\overrightarrow{F}(t)=m\overrightarrow{a}(t)=-m\omega^2(a\cos \omega t \hat{\bold{i}} + a \sin \omega t \hat{\bold{j}})$

\end{enumerate}

\end{document}
