%%%%%%%%%%%%%%%%%%%%%%%%%%%%%%%%%%%%%%%%%%%%%%%%%%%%%%%%%%%%%%%%%%%%%%%%%%%%%%%%%%%%%%%%%%%%%%%%%%%%%%%%%%%%%%%%%%%%%%%%%%%%%%%%%%%%%%%%%%%%%%%%%%%%%%%%%%%%%%%%%%%%%%%%%%%%%%%%%%%%%%%%%%%%
% Written By Michael Brodskiy
% Class: Analytic Geometry & Calculus III (Math-292)
% Professor: V. Cherkassky
%%%%%%%%%%%%%%%%%%%%%%%%%%%%%%%%%%%%%%%%%%%%%%%%%%%%%%%%%%%%%%%%%%%%%%%%%%%%%%%%%%%%%%%%%%%%%%%%%%%%%%%%%%%%%%%%%%%%%%%%%%%%%%%%%%%%%%%%%%%%%%%%%%%%%%%%%%%%%%%%%%%%%%%%%%%%%%%%%%%%%%%%%%%%

\documentclass[12pt]{article} 
\usepackage{alphalph}
\usepackage[utf8]{inputenc}
\usepackage[russian,english]{babel}
\usepackage{titling}
\usepackage{amsmath}
\usepackage{graphicx}
\usepackage{enumitem}
\usepackage{amssymb}
\usepackage[super]{nth}
\usepackage{everysel}
\usepackage{ragged2e}
\usepackage{geometry}
\usepackage{fancyhdr}
\geometry{top=1.0in,bottom=1.0in,left=1.0in,right=1.0in}
\newcommand{\subtitle}[1]{%
  \posttitle{%
    \par\end{center}
    \begin{center}\large#1\end{center}
    \vskip0.5em}%

}
\usepackage{hyperref}
\hypersetup{
colorlinks=true,
linkcolor=blue,
filecolor=magenta,      
urlcolor=blue,
citecolor=blue,
}

\urlstyle{same}


\title{Lecture VIII Notes}
\date{\today}
\author{Michael Brodskiy\\ \small Professor: V. Cherkassky}

% Mathematical Operations:

% Sum: $$\sum_{n=a}^{b} f(x) $$
% Integral: $$\int_{lower}^{upper} f(x) dx$$
% Limit: $$\lim_{x\to\infty} f(x)$$

\begin{document}

\maketitle

\section{Projectile Motion $-$ 13.4}

Given an acceleration, $\overrightarrow{a}(t)$, one may find the velocity and position formulas. In projectile motion, the acceleration is always $\overrightarrow{a}(t) = g = 9.80665\left[\frac{m}{s^2}\right]$\\

From here, one may find that: $\Delta v = \int \overrightarrow{a}(t) dt \Longrightarrow v-v_o=\overrightarrow{a}(t)t\Longrightarrow \overrightarrow{v}(t)=v_o+\overrightarrow{a}(t)t$\\

Furthermore, one may find the position vector by integrating once more: $\Delta r = \int_{}^{}  v_o + \overrightarrow{a}(t)t dt \Longrightarrow r-r_o=v_ot+\frac{1}{2}\overrightarrow{a}(t)t^2\Longrightarrow \overrightarrow{r}(t)=r_o + v_ot+\frac{1}{2}\overrightarrow{a}(t)t^2$\\

In projectile motion, acceleration only acts in the vertical direction, and, therefore:
\begin{tabular}{c}

  $\overrightarrow{r}_y(t)=r_{oy}+v_{oy}t-\frac{1}{2}gt^2$ \\
  $\overrightarrow{v}_y(t)=v_{oy}+gt$ \\
  $\overrightarrow{r}_x(t)=r_{ox}+v_{ox}t$ \\
  $\overrightarrow{v}_x(t)=v_{ox}$

    
\end{tabular}

One may find the acceleration through a different method by using the curvature, $\kappa$. $\kappa = \frac{|\overrightarrow{T}'(t)|}{|\overrightarrow{r}'(t)|}\Longrightarrow |\overrightarrow{T}'(t)|=\kappa |\overrightarrow{v}(t)|$\\
Simplifying this leaves us with: $\overrightarrow{a}=|\overrightarrow{v}'|\overrightarrow{T} + \kappa |\overrightarrow{v}|^2 \overrightarrow{N}$\\
This makes sense, because on a straight line, $\kappa = 0$, and, therefore, on a straight line, $\overrightarrow{a}=|\overrightarrow{v}'|\overrightarrow{T}$\\

The components of acceleration, $a_N$ and $a_T$ can be found using the formulas: $a_N=\frac{|\overrightarrow{r}'(t)\text{ x } \overrightarrow{r}''(t)|}{|\overrightarrow{r}'(t)|}$ and $a_T = \frac{\overrightarrow{r}'(t)\cdot \overrightarrow{r}''(t)}{|\overrightarrow{r}'(t)|}$

\end{document}
