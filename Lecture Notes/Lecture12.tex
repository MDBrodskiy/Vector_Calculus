%%%%%%%%%%%%%%%%%%%%%%%%%%%%%%%%%%%%%%%%%%%%%%%%%%%%%%%%%%%%%%%%%%%%%%%%%%%%%%%%%%%%%%%%%%%%%%%%%%%%%%%%%%%%%%%%%%%%%%%%%%%%%%%%%%%%%%%%%%%%%%%%%%%%%%%%%%%%%%%%%%%%%%%%%%%%%%%%%%%%%%%%%%%%
% Written By Michael Brodskiy
% Class: Analytic Geometry & Calculus III (Math-292)
% Professor: V. Cherkassky
%%%%%%%%%%%%%%%%%%%%%%%%%%%%%%%%%%%%%%%%%%%%%%%%%%%%%%%%%%%%%%%%%%%%%%%%%%%%%%%%%%%%%%%%%%%%%%%%%%%%%%%%%%%%%%%%%%%%%%%%%%%%%%%%%%%%%%%%%%%%%%%%%%%%%%%%%%%%%%%%%%%%%%%%%%%%%%%%%%%%%%%%%%%%

\documentclass[12pt]{article} 
\usepackage{alphalph}
\usepackage[utf8]{inputenc}
\usepackage[russian,english]{babel}
\usepackage{titling}
\usepackage{amsmath}
\usepackage{graphicx}
\usepackage{enumitem}
\usepackage{amssymb}
\usepackage[super]{nth}
\usepackage{everysel}
\usepackage{ragged2e}
\usepackage{geometry}
\usepackage{fancyhdr}
\geometry{top=1.0in,bottom=1.0in,left=1.0in,right=1.0in}
\newcommand{\subtitle}[1]{%
  \posttitle{%
    \par\end{center}
    \begin{center}\large#1\end{center}
    \vskip0.5em}%

}
\usepackage{hyperref}
\hypersetup{
colorlinks=true,
linkcolor=blue,
filecolor=magenta,      
urlcolor=blue,
citecolor=blue,
}

\urlstyle{same}


\title{Lecture XI Notes}
\date{\today}
\author{Michael Brodskiy\\ \small Professor: V. Cherkassky}

% Mathematical Operations:

% Sum: $$\sum_{n=a}^{b} f(x) $$
% Integral: $$\int_{lower}^{upper} f(x) dx$$
% Limit: $$\lim_{x\to\infty} f(x)$$

\begin{document}

\maketitle

\section{The Multivariable Chain Rule $-$ 14.5}

In three dimensions, the formula for the chain rule looks as follows:

$$\frac{dz}{dt}=\frac{\partial f}{\partial x}\frac{dx}{dt} + \frac{\partial f}{\partial y}\frac{dy}{dt}$$

Conditions $z=f(x,y); x=x(t), y=y(t)$\\

\textit{Example:}

$$z=x^2y+3xy^4; x=\sin2t,y=\cos t$$
$$\frac{dz}{dt}=(2xy+3y^4)(2\cos2t)+(x^2+12xy^3)(-\sin t)$$

Another case looks as follows:

If $z=f(x,y); x=g(s,t), y=h(s,t)$, then:

$$\frac{\partial z}{\partial s}=\frac{\partial z}{\partial x}\frac{\partial x}{\partial s} + \frac{\partial z}{\partial y}\frac{\partial y}{\partial s}$$

$$\frac{\partial z}{\partial t}=\frac{\partial z}{\partial x}\frac{\partial x}{\partial t} + \frac{\partial z}{\partial y}\frac{\partial y}{\partial t}$$

The general case of the chain rule, where $u$ is a multivariable function of $n$ variables $x_1$, $x_2$,\dots $x_n$, and each $x$ is a function of $m$ variables $t_1$,\dots$t_m$, then:

$$\frac{\partial u}{\partial t_i}=\frac{\partial u}{\partial x_1}\frac{\partial x_1}{\partial t_i}+\frac{\partial u}{\partial x_2}\frac{\partial x_2}{\partial t_i}+\dots+\frac{\partial u}{\partial x_n}\frac{\partial x_n}{\partial t_i}$$

\subsection{Implicit Differentiation}

If $\frac{\partial f}{\partial y}\neq 0$:

$$\frac{dy}{dx}=-\frac{\frac{\partial f}{\partial x}}{\frac{\partial f}{\partial y}}$$

In addition to this:

$$\frac{\partial z}{\partial x}=-\frac{\frac{\partial f}{\partial x}}{\frac{\partial f}{\partial z}}$$

and

$$\frac{\partial z}{\partial y}=-\frac{\frac{\partial f}{\partial y}}{\frac{\partial f}{\partial z}}$$

\section{The Directional Derivative $-$ 14.6}

The definition of a directional derivative is:

$$ D_{\hat{u}}=\lim_{h\to0} \frac{f(x_o+ha, y_o + hb)-f(x_o,y_o)}{h} $$

In more simplified terms:

$$D_{\hat{u}}=\frac{\partial f}{\partial x} a + \frac{\partial f}{\partial y} b$$

\subsection{The Gradient Vector}

The gradient vector is written $\overrightarrow{\nabla} f$, and:

$$\overrightarrow{\nabla} f = \frac{\partial f}{\partial x}\hat{\bold{i}} + \frac{\partial f}{\partial y}\hat{\bold{j}}$$

In the case that $u = f(x,y,z)$:

$$\overrightarrow{\nabla} f = \frac{\partial f}{\partial x}\hat{\bold{i}} + \frac{\partial f}{\partial y}\hat{\bold{j}}+\frac{\partial f}{\partial z}\hat{\bold{k}}$$

and

$$D_{\hat{u}}=\overrightarrow{\nabla}f(x,y,z)\cdot\hat{u}$$

The greatest value of the gradient, written as $|\overrightarrow{\nabla}f|$, occurs in the same direction for $\hat{\bold{u}}$ and $\overrightarrow{\nabla}f$

\end{document}
