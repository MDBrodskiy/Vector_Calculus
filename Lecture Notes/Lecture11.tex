%%%%%%%%%%%%%%%%%%%%%%%%%%%%%%%%%%%%%%%%%%%%%%%%%%%%%%%%%%%%%%%%%%%%%%%%%%%%%%%%%%%%%%%%%%%%%%%%%%%%%%%%%%%%%%%%%%%%%%%%%%%%%%%%%%%%%%%%%%%%%%%%%%%%%%%%%%%%%%%%%%%%%%%%%%%%%%%%%%%%%%%%%%%%
% Written By Michael Brodskiy
% Class: Analytic Geometry & Calculus III (Math-292)
% Professor: V. Cherkassky
%%%%%%%%%%%%%%%%%%%%%%%%%%%%%%%%%%%%%%%%%%%%%%%%%%%%%%%%%%%%%%%%%%%%%%%%%%%%%%%%%%%%%%%%%%%%%%%%%%%%%%%%%%%%%%%%%%%%%%%%%%%%%%%%%%%%%%%%%%%%%%%%%%%%%%%%%%%%%%%%%%%%%%%%%%%%%%%%%%%%%%%%%%%%

\documentclass[12pt]{article} 
\usepackage{alphalph}
\usepackage[utf8]{inputenc}
\usepackage[russian,english]{babel}
\usepackage{titling}
\usepackage{amsmath}
\usepackage{graphicx}
\usepackage{enumitem}
\usepackage{amssymb}
\usepackage[super]{nth}
\usepackage{everysel}
\usepackage{ragged2e}
\usepackage{geometry}
\usepackage{fancyhdr}
\geometry{top=1.0in,bottom=1.0in,left=1.0in,right=1.0in}
\newcommand{\subtitle}[1]{%
  \posttitle{%
    \par\end{center}
    \begin{center}\large#1\end{center}
    \vskip0.5em}%

}
\usepackage{hyperref}
\hypersetup{
colorlinks=true,
linkcolor=blue,
filecolor=magenta,      
urlcolor=blue,
citecolor=blue,
}

\urlstyle{same}


\title{Lecture XI Notes}
\date{\today}
\author{Michael Brodskiy\\ \small Professor: V. Cherkassky}

% Mathematical Operations:

% Sum: $$\sum_{n=a}^{b} f(x) $$
% Integral: $$\int_{lower}^{upper} f(x) dx$$
% Limit: $$\lim_{x\to\infty} f(x)$$

\begin{document}

\maketitle

\subsection{Second Order Partial Derivatives}

Although the order of second partial differentiation usually does not matter, sometimes the second order partial derivatives may not be equal. Differentiating $\frac{\partial f}{\partial x}$ by $\frac{\partial}{\partial y}$ results in $\frac{\partial^2f}{\partial y \partial x}$. If one differentiates in the reversed order, this yields $\frac{\partial^2f}{\partial x \partial y}$, where the order of partial differentiation is read from right to left.\\

\textit{Example Where $\frac{\partial^2f}{\partial y \partial x}=\frac{\partial^2f}{\partial x \partial y}$:}

$$f(x,y)=x^3+x^2y^3-2y^2$$
$$\frac{\partial f}{\partial x}=3x^2+2xy^3$$
$$\frac{\partial f}{\partial y}=3x^2y^2-4y$$
$$\frac{\partial^2f}{\partial y \partial x}=6xy^2$$
$$\frac{\partial^2f}{\partial x \partial y}=6xy^2$$
\begin{center}$\therefore \frac{\partial^2f}{\partial x \partial y}=\frac{\partial^2f}{\partial y \partial x}$\footnote{In most cases, $\frac{\partial^2f}{\partial x \partial y}=\frac{\partial^2f}{\partial y \partial x}$}\end{center}

\subsection{Clairaut's Theorem}

If $\frac{\partial^2f}{\partial x \partial y}$ and $\frac{\partial^2f}{\partial y \partial x}$ are continuous at point $(a, b)$, then $\frac{\partial^2f}{\partial x \partial y}=\frac{\partial^2f}{\partial y \partial x}$ at $(a, b)$\footnote{This rule applies to any nth order derivative}

\subsection{Laplace's Differential Equations}

Named after Pierre Laplace, the partial differential equation looks like this:

$$\frac{\partial^2u}{\partial x^2} + \frac{\partial^2u}{\partial y^2}=0$$

Any function that fits the criteria is called a \bold{harmonic function}

\subsection{Wave Equations}

Any wave, whose displacement is defined by the function $u(x,t)$ must satisfy the wave equation:

$$\frac{\partial^2u}{\partial t^2}=a^2\frac{\partial^2u}{\partial x^2}$$

\section{Tangent Plane Approximation and Total Differentials $-$ 14.4}

The tangent plane approximation function is given by:

$$z-z_o=f_x(x_o,y_o)(x-x_o)+f_y(x_o,y_o)(y-y_o)$$

The definition of a total differential is:

$$dz = \frac{\partial z}{\partial x}dx + \frac{\partial z}{\partial y}dy$$

$dz$ represents the change in the plane approximation

\end{document}
