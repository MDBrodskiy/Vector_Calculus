%%%%%%%%%%%%%%%%%%%%%%%%%%%%%%%%%%%%%%%%%%%%%%%%%%%%%%%%%%%%%%%%%%%%%%%%%%%%%%%%%%%%%%%%%%%%%%%%%%%%%%%%%%%%%%%%%%%%%%%%%%%%%%%%%%%%%%%%%%%%%%%%%%%%%%%%%%%%%%%%%%%%%%%%%%%%%%%%%%%%%%%%%%%%
% Written By Michael Brodskiy
% Class: Analytic Geometry & Calculus III (Math-292)
% Professor: V. Cherkassky
%%%%%%%%%%%%%%%%%%%%%%%%%%%%%%%%%%%%%%%%%%%%%%%%%%%%%%%%%%%%%%%%%%%%%%%%%%%%%%%%%%%%%%%%%%%%%%%%%%%%%%%%%%%%%%%%%%%%%%%%%%%%%%%%%%%%%%%%%%%%%%%%%%%%%%%%%%%%%%%%%%%%%%%%%%%%%%%%%%%%%%%%%%%%

\documentclass[12pt]{article} 
\usepackage{alphalph}
\usepackage[utf8]{inputenc}
\usepackage[russian,english]{babel}
\usepackage{titling}
\usepackage{amsmath}
\usepackage{graphicx}
\usepackage{enumitem}
\usepackage{amssymb}
\usepackage[super]{nth}
\usepackage{everysel}
\usepackage{ragged2e}
\usepackage{geometry}
\usepackage{fancyhdr}
\geometry{top=1.0in,bottom=1.0in,left=1.0in,right=1.0in}
\newcommand{\subtitle}[1]{%
  \posttitle{%
    \par\end{center}
    \begin{center}\large#1\end{center}
    \vskip0.5em}%

}
\usepackage{hyperref}
\hypersetup{
colorlinks=true,
linkcolor=blue,
filecolor=magenta,      
urlcolor=blue,
citecolor=blue,
}

\urlstyle{same}


\title{Lecture XVI Notes}
\date{\today}
\author{Michael Brodskiy\\ \small Professor: V. Cherkassky}

% Mathematical Operations:

% Sum: $$\sum_{n=a}^{b} f(x) $$
% Integral: $$\int_{lower}^{upper} f(x) dx$$
% Limit: $$\lim_{x\to\infty} f(x)$$

\begin{document}

\maketitle

In polar coordinates, equations relating $x$ and $y$ to $r(\theta)$ must be defined:

\begin{center}
\begin{tabular}{|c c c|}
    \hline
    & & \\
$x^2+y^2=r^2$ & $x=r\cos\theta$ & $y=r\sin\theta$\\
    & & \\
    \hline
\end{tabular}
\end{center}

On the region, $R$, defined by:

$$R=\{(r,\theta)|a\leq r\leq b, \alpha\leq \theta\leq \beta\}$$

The double integral for the area of a polar region is as follows:

$$\iint_R f(x,y)\,dA=\lim_{(n,m)\to\infty} \sum_{i=1}^m\sum_{j=1}^n f(r_i^*\cos\theta_j^*,r_i^*\sin\theta_j^*)\,\DeltaA$$

This all simplifies down to:

$$\int_\alpha^\beta\int_a^b f(r\cos\theta,r\sin\theta)r\, dr\, d\theta$$

For general polar regions, the same process as for Cartesian coordinates is used:

$$\iint_D f(x,y)\,dA=\int_\alpha^\beta\int_{h_1(\theta)}^{h_2(\theta)} f(r\cos\theta,r\sin\theta)r\,dr\,d\theta$$

\end{document}
