%%%%%%%%%%%%%%%%%%%%%%%%%%%%%%%%%%%%%%%%%%%%%%%%%%%%%%%%%%%%%%%%%%%%%%%%%%%%%%%%%%%%%%%%%%%%%%%%%%%%%%%%%%%%%%%%%%%%%%%%%%%%%%%%%%%%%%%%%%%%%%%%%%%%%%%%%%%%%%%%%%%%%%%%%%%%%%%%%%%%%%%%%%%%
% Written By Michael Brodskiy
% Class: Analytic Geometry & Calculus III (Math-292)
% Professor: V. Cherkassky
%%%%%%%%%%%%%%%%%%%%%%%%%%%%%%%%%%%%%%%%%%%%%%%%%%%%%%%%%%%%%%%%%%%%%%%%%%%%%%%%%%%%%%%%%%%%%%%%%%%%%%%%%%%%%%%%%%%%%%%%%%%%%%%%%%%%%%%%%%%%%%%%%%%%%%%%%%%%%%%%%%%%%%%%%%%%%%%%%%%%%%%%%%%%

\documentclass[12pt]{article} 
\usepackage{alphalph}
\usepackage[utf8]{inputenc}
\usepackage[russian,english]{babel}
\usepackage{titling}
\usepackage{amsmath}
\usepackage{graphicx}
\usepackage{enumitem}
\usepackage{amssymb}
\usepackage[super]{nth}
\usepackage{everysel}
\usepackage{ragged2e}
\usepackage{geometry}
\usepackage{fancyhdr}
\geometry{top=1.0in,bottom=1.0in,left=1.0in,right=1.0in}
\newcommand{\subtitle}[1]{%
  \posttitle{%
    \par\end{center}
    \begin{center}\large#1\end{center}
    \vskip0.5em}%

}
\usepackage{hyperref}
\hypersetup{
colorlinks=true,
linkcolor=blue,
filecolor=magenta,      
urlcolor=blue,
citecolor=blue,
}

\urlstyle{same}


\title{Lecture X Notes}
\date{\today}
\author{Michael Brodskiy\\ \small Professor: V. Cherkassky}

% Mathematical Operations:

% Sum: $$\sum_{n=a}^{b} f(x) $$
% Integral: $$\int_{lower}^{upper} f(x) dx$$
% Limit: $$\lim_{x\to\infty} f(x)$$

\begin{document}

\maketitle

\section{Limits of Multivariable Functions $-$ 14.2}

Just like in Calculus I, there are three conditions for a function to be continuous:

\begin{enumerate}

  \item $f(a)$ exists

  \item $\lim_{x\to a} f(x) = f(a)$

  \item $\lim_{x\to a^-} f(x) =\lim_{x\to a^+} f(x)$

\end{enumerate}

In three dimensions, the same rules apply, in addition to a new one:

$$\lim_{(x,y)\to(a,b)}f(x,y)=L$$ if, for every number $\epsilon > 0$ there is a corresponding number $\partial > 0$ such that,\\ if $(x,y)\in D$ and $0 < \sqrt{(x-a)^2+(y-b)^2} < \partial$ then $|f(x,y) - L| < \epsilon$

\section{Partial Derivatives $-$ 14.3}

For the function, $z=f(x,y)$, the partial derivatives may be expressed through the use of the limit definition of a derivative, where:

$$f_x(x,y)=\lim_{\Delta x\to0} \frac{f(x+\Delta x,y)-f(x,y)}{\Delta x}$$
$$f_y(x,y)=\lim_{\Delta y\to0} \frac{f(x,y + \Delta y)-f(x,y)}{\Delta y}$$

Some common notations are:

$$f_x(x,y)=f_x=\frac{\partial f}{\partial x}=\frac{\partial}{\partial x}f(x,y)=\frac{\partial z}{\partial x} = f_1=D_1f=D_xf$$
$$f_y(x,y)=f_y=\frac{\partial f}{\partial y}=\frac{\partial}{\partial y}f(x,y)=\frac{\partial z}{\partial y} = f_2=D_2f=D_yf$$

\subsection{Finding the Partial Derivative}

To find the partial derivatives of the function $f(x,y)$, one would hold $y$ constant, while differentiating with respect to $x$ to find $\frac{\partial f}{\partial x}$, and hold $x$ constant, while differentiating with respect to $y$ to find $\frac{\partial f}{\partial y}$


\end{document}
