%%%%%%%%%%%%%%%%%%%%%%%%%%%%%%%%%%%%%%%%%%%%%%%%%%%%%%%%%%%%%%%%%%%%%%%%%%%%%%%%%%%%%%%%%%%%%%%%%%%%%%%%%%%%%%%%%%%%%%%%%%%%%%%%%%%%%%%%%%%%%%%%%%%%%%%%%%%%%%%%%%%%%%%%%%%%%%%%%%%%%%%%%%%%
% Written By Michael Brodskiy
% Class: Analytic Geometry & Calculus III (Math-292)
% Professor: V. Cherkassky
%%%%%%%%%%%%%%%%%%%%%%%%%%%%%%%%%%%%%%%%%%%%%%%%%%%%%%%%%%%%%%%%%%%%%%%%%%%%%%%%%%%%%%%%%%%%%%%%%%%%%%%%%%%%%%%%%%%%%%%%%%%%%%%%%%%%%%%%%%%%%%%%%%%%%%%%%%%%%%%%%%%%%%%%%%%%%%%%%%%%%%%%%%%%

\documentclass[12pt]{article} 
\usepackage{alphalph}
\usepackage[utf8]{inputenc}
\usepackage[russian,english]{babel}
\usepackage{titling}
\usepackage{amsmath}
\usepackage{graphicx}
\usepackage{enumitem}
\usepackage{amssymb}
\usepackage[super]{nth}
\usepackage{everysel}
\usepackage{ragged2e}
\usepackage{geometry}
\usepackage{fancyhdr}
\geometry{top=1.0in,bottom=1.0in,left=1.0in,right=1.0in}
\newcommand{\subtitle}[1]{%
  \posttitle{%
    \par\end{center}
    \begin{center}\large#1\end{center}
    \vskip0.5em}%

}
\usepackage{hyperref}
\hypersetup{
colorlinks=true,
linkcolor=blue,
filecolor=magenta,      
urlcolor=blue,
citecolor=blue,
}

\urlstyle{same}


\title{Lecture XIII Notes}
\date{\today}
\author{Michael Brodskiy\\ \small Professor: V. Cherkassky}

% Mathematical Operations:

% Sum: $$\sum_{n=a}^{b} f(x) $$
% Integral: $$\int_{lower}^{upper} f(x) dx$$
% Limit: $$\lim_{x\to\infty} f(x)$$

\begin{document}

\maketitle

\section{Maxima and Minima $-$ 14.7}


In three dimensions, a point is determined to be critical if:
$$\frac{\partial f}{\partial x}(a,b)=0,DNE\text{ and } \frac{\partial f}{\partial y}(a,b)=0,DNE$$

From here, it needs to be determined whether or not there is an extrema at point $(a,b)$, and whether it is a minima or maxima.

\subsection{The Second Derivative Test}

First, $\frac{\partial^2f}{\partial x^2}\text{, } \frac{\partial^2f}{\partial x \partial y}\text{, } \frac{\partial^2f}{\partial y \partial x}\text{ and } \frac{\partial^2f}{\partial y^2}$ must be found. Then the determinant, $D$, of these must be found:
\begin{center}
  \begin{tabular}{|c c|}
    $f_{xx}$ & $f_{xy}$ \\
    $f_{yx}$ & $f_{yy}$ \\

  \end{tabular}
\end{center}

$$f_{xx}f_{yy}-(f_{xy})^2$$

\begin{enumerate}

  \item If $D > 0$, and $f_{xx}(a,b)>0$, then $f(a,b)$ is a local minimum

  \item If $D > 0$, and $f_{xx}(a,b)<0$, then $f(a,b)$ is a local maximum

  \item If $D < 0$, then $f(a,b)$ is not a local maximum or minimum (saddle point)

\subsection{Bounded Sets}

In a closed, bounded set, $f$ attains a maximum at $(x_1,y_1)$, and a minimum at $(x_2,y_2)$


\end{enumerate}

\end{document}
