%%%%%%%%%%%%%%%%%%%%%%%%%%%%%%%%%%%%%%%%%%%%%%%%%%%%%%%%%%%%%%%%%%%%%%%%%%%%%%%%%%%%%%%%%%%%%%%%%%%%%%%%%%%%%%%%%%%%%%%%%%%%%%%%%%%%%%%%%%%%%%%%%%%%%%%%%%%%%%%%%%%%%%%%%%%%%%%%%%%%%%%%%%%%
% Written By Michael Brodskiy
% Class: Analytic Geometry & Calculus III (Math-292)
% Professor: V. Cherkassky
%%%%%%%%%%%%%%%%%%%%%%%%%%%%%%%%%%%%%%%%%%%%%%%%%%%%%%%%%%%%%%%%%%%%%%%%%%%%%%%%%%%%%%%%%%%%%%%%%%%%%%%%%%%%%%%%%%%%%%%%%%%%%%%%%%%%%%%%%%%%%%%%%%%%%%%%%%%%%%%%%%%%%%%%%%%%%%%%%%%%%%%%%%%%

\documentclass[12pt]{article} 
\usepackage{alphalph}
\usepackage[utf8]{inputenc}
\usepackage[russian,english]{babel}
\usepackage{titling}
\usepackage{amsmath}
\usepackage{graphicx}
\usepackage{enumitem}
\usepackage{amssymb}
\usepackage[super]{nth}
\usepackage{everysel}
\usepackage{ragged2e}
\usepackage{geometry}
\usepackage{fancyhdr}
\geometry{top=1.0in,bottom=1.0in,left=1.0in,right=1.0in}
\newcommand{\subtitle}[1]{%
  \posttitle{%
    \par\end{center}
    \begin{center}\large#1\end{center}
    \vskip0.5em}%

}
\usepackage{hyperref}
\hypersetup{
colorlinks=true,
linkcolor=blue,
filecolor=magenta,      
urlcolor=blue,
citecolor=blue,
}

\urlstyle{same}


\title{Lecture XIV Notes}
\date{\today}
\author{Michael Brodskiy\\ \small Professor: V. Cherkassky}

% Mathematical Operations:

% Sum: $$\sum_{n=a}^{b} f(x) $$
% Integral: $$\int_{lower}^{upper} f(x) dx$$
% Limit: $$\lim_{x\to\infty} f(x)$$

\begin{document}

\maketitle

\section{Multiple Integrals $-$ 15.1}

In calculus I, integrals were defined as the area, $A$, under a curve, where:

$$A=\lim_{x\to\infty}\sum_{i=1}^n f(x_i^*) \Delta x$$

This formula was simplified to what is known as the integral:

$$\int_a^b f(x)\,dx$$

The volume of a multivariable function, then, is given by:

$$V \approx \sum_{i=1}^m \sum_{j=1}^n f(x_{ij}^*, y_{ij}^*) \Delta A$$
$$V=\lim_{m,n\to\infty}\sum_{i=1}^m \sum_{j=1}^n f(x_{ij}^*, y_{ij}^*) \Delta A $$
$$V=\iint_R f(x,y)\,dA$$

\subsection{Double Integral Properties}

Double integrals have many of the same properties as single integrals:

$$1. \iint_R [f(x,y)+g(x,y)]\,dA=\iint_R f(x,y)\,dA + \iint_R g(x,y)\,dA$$

If $c$ is a constant:

$$\iint_R cf(x,y)\,dA=c\iint_R f(x,y)\,dA$$

If $f(x,y)\geq g(x,y)$ for all $(x,y)$, then:

$$\iint_R f(x,y)\,dA \geq \iint_R g(x,y)\,dA$$

If $g(x)$ is a function of $x$ and $h(y)$ is a function of $y$:

$$\iint_R g(x)h(y) dA=\int_a^b g(x)\,dx\int_c^d h(y)\,dy$$

\subsection{Iterated Integrals}

The double integral, $\iint_R f(x,y)\,dA$, may be broken up in order to be calculated with ease:

$$\iint_R f(x,y) dA=\int_a^b \int_c^d f(x,y)\,dy\,dx$$

This may be solved using the method of partial integrals, where, much like partial derivatives, only one of the variables is treated as a variable at once. As a result of this, just like with partial derivatives, the order that the operation is done in yields the same result no matter what:

$$\int_a^b \int_c^d f(x,y)\,dy\,dx=\int_c^d\int_a^b f(x,y)\,dx\,dy$$

\subsection{Fubini's Theorem}

If $f(x,y)$ is continuous on $R=\{(x,y)|a\leq x\leq b, c\leq y\leq d\}$

$$\iint_R f(x,y) dA=\int_a^b \int_c^d f(x,y)\,dy\,dx=\int_c^d\int_a^b f(x,y)\,dx\,dy$$

\end{document}
