%%%%%%%%%%%%%%%%%%%%%%%%%%%%%%%%%%%%%%%%%%%%%%%%%%%%%%%%%%%%%%%%%%%%%%%%%%%%%%%%%%%%%%%%%%%%%%%%%%%%%%%%%%%%%%%%%%%%%%%%%%%%%%%%%%%%%%%%%%%%%%%%%%%%%%%%%%%%%%%%%%%%%%%%%%%%%%%%%%%%%%%%%%%%
% Written By Michael Brodskiy
% Class: Analytic Geometry & Calculus III (Math-292)
% Professor: V. Cherkassky
%%%%%%%%%%%%%%%%%%%%%%%%%%%%%%%%%%%%%%%%%%%%%%%%%%%%%%%%%%%%%%%%%%%%%%%%%%%%%%%%%%%%%%%%%%%%%%%%%%%%%%%%%%%%%%%%%%%%%%%%%%%%%%%%%%%%%%%%%%%%%%%%%%%%%%%%%%%%%%%%%%%%%%%%%%%%%%%%%%%%%%%%%%%%

\documentclass[12pt]{article} 
\usepackage{alphalph}
\usepackage[utf8]{inputenc}
\usepackage[russian,english]{babel}
\usepackage{titling}
\usepackage{amsmath}
\usepackage{graphicx}
\usepackage{enumitem}
\usepackage{amssymb}
\usepackage[super]{nth}
\usepackage{everysel}
\usepackage{ragged2e}
\usepackage{geometry}
\usepackage{fancyhdr}
\geometry{top=1.0in,bottom=1.0in,left=1.0in,right=1.0in}
\newcommand{\subtitle}[1]{%
  \posttitle{%
    \par\end{center}
    \begin{center}\large#1\end{center}
    \vskip0.5em}%

}
\usepackage{hyperref}
\hypersetup{
colorlinks=true,
linkcolor=blue,
filecolor=magenta,      
urlcolor=blue,
citecolor=blue,
}

\urlstyle{same}


\title{Lecture I Notes}
\date{\today}
\author{Michael Brodskiy\\ \small Professor: V. Cherkassky}


\begin{document}

\maketitle


\section{Chapter 12}

\subsection{Vectors (12.1)}

\begin{flushleft}
Basic Vector manipulations:
\end{flushleft}
\underline{Multiplication or Division}: some vector, $\vec{a}$, multiplied or divided by a constant, $k$, is simply the magnitude (length) of the vector multiplied or divided by the scalar\\
\underline{Addition or Subtraction}: The addition or subtraction of two vectors forms a triangle. Finding the third side of the triangle will give you the magnitude (length) of the resulting vector.\\

$$\text{The distance between to points, or the magnitude of the vector may be found using the distance formula:} \sqrt{(x_1-x_2)+(y_1-y_2)} \text{Or, in three dimensions:} \sqrt{(x_1-x_2)+(y_1-y_2)+(z_1-z_2)}$$

$$\text{These equations come from the circular or spherical formula for radius: }(x-h)^2+(y-k)^2=r^2, \text{and } (x-h)^2+(y-k)^2+(z-c)^2=r^2 $$

\textit{Practice Problem:\\What is the radius and center of the figure?}

$$x^2+y^2+z^2-2x-4y+8z=15\\x^2-2x+y^2-4y+z^2+8z=15\\(x-1)^2+(y-2)^2+(z+4)^2=36$$
Sphere with center at $(1, 2, -4)$, and radius 6




\end{document}
