%%%%%%%%%%%%%%%%%%%%%%%%%%%%%%%%%%%%%%%%%%%%%%%%%%%%%%%%%%%%%%%%%%%%%%%%%%%%%%%%%%%%%%%%%%%%%%%%%%%%%%%%%%%%%%%%%%%%%%%%%%%%%%%%%%%%%%%%%%%%%%%%%%%%%%%%%%%%%%%%%%%%%%%%%%%%%%%%%%%%%%%%%%%%
% Written By Michael Brodskiy
% Class: Analytic Geometry & Calculus III (Math-292)
% Professor: V. Cherkassky
%%%%%%%%%%%%%%%%%%%%%%%%%%%%%%%%%%%%%%%%%%%%%%%%%%%%%%%%%%%%%%%%%%%%%%%%%%%%%%%%%%%%%%%%%%%%%%%%%%%%%%%%%%%%%%%%%%%%%%%%%%%%%%%%%%%%%%%%%%%%%%%%%%%%%%%%%%%%%%%%%%%%%%%%%%%%%%%%%%%%%%%%%%%%

\documentclass[12pt]{article} 
\usepackage{alphalph}
\usepackage[utf8]{inputenc}
\usepackage[russian,english]{babel}
\usepackage{titling}
\usepackage{amsmath}
\usepackage{graphicx}
\usepackage{enumitem}
\usepackage{amssymb}
\usepackage[super]{nth}
\usepackage{everysel}
\usepackage{ragged2e}
\usepackage{geometry}
\usepackage{fancyhdr}
\usepackage{cancel}
\geometry{top=1.0in,bottom=1.0in,left=1.0in,right=1.0in}
\newcommand{\subtitle}[1]{%
  \posttitle{%
    \par\end{center}
    \begin{center}\large#1\end{center}
    \vskip0.5em}%

}
\usepackage{hyperref}
\hypersetup{
colorlinks=true,
linkcolor=blue,
filecolor=magenta,      
urlcolor=blue,
citecolor=blue,
}

\urlstyle{same}


\title{Lecture XXI Notes}
\date{\today}
\author{Michael Brodskiy\\ \small Professor: V. Cherkassky}

% Mathematical Operations:

% Sum: $$\sum_{n=a}^{b} f(x) $$
% Integral: $$\int_{lower}^{upper} f(x) dx$$
% Limit: $$\lim_{x\to\infty} f(x)$$

\begin{document}

\maketitle

\section{Fundamental Theorem of Line Integrals $-$ 16.3}

The Fundamental Theorem of Line Integrals states that, if $C$ is a smooth, continuous curve, and $\overrightarrow{r}'(t)$, $a\leq t\leq b$, then:

$$\int_C \nabla f\,dr=f(\overrightarrow{r}(b))-f(\overrightarrow{r}(a))$$

\textit{Example:} Given $f(x,y,z)=\cos(\pi x)+\sin(\pi y)-xyz$, and $C$ is any path from $(1,\frac{1}{2}, 2)$ to $(2,1,-1)$, Find:

$$\int_C \nabla f\,dr$$

$$\int_C \nabla f\,dr=f(2,1,-1)-f(1,\frac{1}{2},2)$$

$$(\cos(2\pi)+\sin(\pi)+2)-(\cos(\pi)+\sin(\frac{\pi}{2})-1)$$

$$\int_C \nabla f\,dr=4$$

Recalling conservative vector fields, where $\overrightarrow{F}=\nabla f \Leftarrow$ potential function  

The Fundamental Theorem of Line Integrals also states that conservative vector fields are independent of path. That is, as long as any path, say $C_1$ and $C_2$, start and end at the same point, then:

$$\int_{C_1} \nabla f\,dr=\int_{C_2} \nabla f\,dr$$

Any path, $C$, that starts at the same point at which it terminates is called closed.

There are two types of closed paths:

\begin{enumerate}

  \item Simple

  \item Non-Simple

\end{enumerate}

A simple path is a shape such as a circle or ellipse. A non-simple path crosses over itself, and example of such a shape being the ``$\infty$'' symbol.

A region, $D$ is called open when it does not contain any of its own boundaries.

Any region, $D$, in which one may connect any two points without exiting the boundaries is known as connected.

Furthermore, a region, $D$ may be called simply connected when there are no holes.

\begin{enumerate}

  \item If $C$ is independent of path:
    $$\int_{C_1}\nabla f\,dr=\int_{C_2}\nabla f\,dr$$

  \item $\overrightarrow{F}$ is conservative if $\overrightarrow{F}=\nabla f$

  \item $\overrightarrow{F}$ is a conservative vector field if $D$ is an open and connected region, where $\int_C \overrightarrow{F}\,dr$ is independent of path (in $D$).

  \item $\int_C \overrightarrow{F}\,dr$ is independent of path in a region, $D$, if $\int_C \overrightarrow{F}\,dr=0$ on every closed path $C$ in $D$.

\end{enumerate}  

    Given a function, $\overrightarrow{F}(x,y)=P(x,y)\bold{\hat{i}}+Q(x,y)\bold{\hat{j}},  $\overrightarrow{F}$ is conservative if:
    $$\frac{\partial P}{\partial y}=\frac{\partial Q}{\partial x}$$

    \subsection{Solving for $\nabla f$}

    Given $\overrightarrow{F}(x,y)=(2x^3y^4+x)\bold{\hat{i}}+(2x^4y^3+y)\bold{\hat{j}}$, determine whether $\overrightarrow{F}$ is a conservative vector field, and, if so, find $\nabla f$

    $$\frac{\partial P}{\partial y}=8x^3y^3$$
    $$\frac{\partial Q}{\partial x}=8x^3y^3$$
    Therefore, it is conservative

    $$\frac{\partial f}{\partial x}=2x^3y^4+x$$
    $$\frac{\partial f}{\partial y}=2x^4y^3+y$$
    $$f(x,y)=\int_C 2x^3y^4+x\,dx\Rightarrow\frac{1}{2}(x^4y^4+x^2)+h(y)$$
    $$\frac{\partial f}{\partial y}=2x^4y^3+h'(y)=2x^4y^3+y$$
    $${2x^4y^3}+h'(y)=\cancel{2x^4y^3}+y$$
    $$h'(y)=y\Rightarrow h(y)=\frac{1}{2}y^2$$
    $$f(x,y)=\frac{1}{2}(x^4y^4+x^2+y^2)$$


\end{document}

