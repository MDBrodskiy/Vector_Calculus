%%%%%%%%%%%%%%%%%%%%%%%%%%%%%%%%%%%%%%%%%%%%%%%%%%%%%%%%%%%%%%%%%%%%%%%%%%%%%%%%%%%%%%%%%%%%%%%%%%%%%%%%%%%%%%%%%%%%%%%%%%%%%%%%%%%%%%%%%%%%%%%%%%%%%%%%%%%%%%%%%%%%%%%%%%%%%%%%%%%%%%%%%%%%
% Written By Michael Brodskiy
% Class: Analytic Geometry & Calculus III (Math-292)
% Professor: V. Cherkassky
%%%%%%%%%%%%%%%%%%%%%%%%%%%%%%%%%%%%%%%%%%%%%%%%%%%%%%%%%%%%%%%%%%%%%%%%%%%%%%%%%%%%%%%%%%%%%%%%%%%%%%%%%%%%%%%%%%%%%%%%%%%%%%%%%%%%%%%%%%%%%%%%%%%%%%%%%%%%%%%%%%%%%%%%%%%%%%%%%%%%%%%%%%%%

\documentclass[12pt]{article} 
\usepackage{alphalph}
\usepackage[utf8]{inputenc}
\usepackage[russian,english]{babel}
\usepackage{titling}
\usepackage{amsmath}
\usepackage{graphicx}
\usepackage{enumitem}
\usepackage{amssymb}
\usepackage[super]{nth}
\usepackage{everysel}
\usepackage{ragged2e}
\usepackage{geometry}
\usepackage{fancyhdr}
\geometry{top=1.0in,bottom=1.0in,left=1.0in,right=1.0in}
\newcommand{\subtitle}[1]{%
  \posttitle{%
    \par\end{center}
    \begin{center}\large#1\end{center}
    \vskip0.5em}%

}
\usepackage{hyperref}
\hypersetup{
colorlinks=true,
linkcolor=blue,
filecolor=magenta,      
urlcolor=blue,
citecolor=blue,
}

\urlstyle{same}


\title{Lecture XIX Notes}
\date{\today}
\author{Michael Brodskiy\\ \small Professor: V. Cherkassky}

% Mathematical Operations:

% Sum: $$\sum_{n=a}^{b} f(x) $$
% Integral: $$\int_{lower}^{upper} f(x) dx$$
% Limit: $$\lim_{x\to\infty} f(x)$$

\begin{document}

\maketitle

\section{Vector Fields $-$ 16.1}

If the region $D$ lies in $\mathbb{R}^2$, then the function for the field is as follows:

$$\bold{\bar{F}}(x,y)=P(x,y)\bold{\hat{i}}+Q(x,y)\bold{\hat{j}}$$

If $D$ lies in $\mathbb{R}^3$:

$$\bold{\bar{F}}(x,y)=P(x,y)\bold{\hat{i}}+Q(x,y)\bold{\hat{j}}+R(x,y)\bold{\hat{k}}$$

The formula for Newton's law of gravitation is as follows:

$$|\bold{\bar{F}}|=\frac{GmM}{r^2}$$

This may be transformed into a vector field by using the following steps:

$$|\bold{\bar{F}}|=\frac{GmM}{r^2}$$
$$|\bold{\bar{F}}|=\frac{GmM}{r^3}\bold{\hat{r}}$$
$$\bold{\hat{r}}=\langle x, y, z\rangle,\,r=\sqrt{x^2+y^2+z^2}$$
$$|\bold{\bar{F}}|=\frac{GmMx\bold{\hat{i}}}{(x^2+y^2+z^2)^{\frac{3}{2}}}+\frac{GmMy\bold{\hat{j}}}{(x^2+y^2+z^2)^{\frac{3}{2}}}+\frac{GmMz\bold{\hat{k}}}{(x^2+y^2+z^2)^{\frac{3}{2}}}$$

Calling back to the gradient, which is written as $\nabla f(x,y)=\frac{\partial f}{\partial x}\bold{\hat{i}} + \frac{\partial f}{\partial y}\bold{\hat{j}}$, where $f(x,y)$ is a scalar function. This may be used in conjunction with vector fields.\\

Any vector field which has $\bold{\bar{F}}=\nabla f$ is called a conservative vector field. In such a case, $f$ is called a potential function for $\bold{\bar{F}}$ 

\textit{In Physics:}
Newton's formula for gravitation is a conservative vector field.
In addition to this, the potential function, $f$, for conservative vector fields was used to find potential electrical and magnetic field, in addition to the aforementioned gravitational field.


\section{Line Integrals $-$ 16.2}

A line integral may be found using the formula:

$$\int_C f(x,y)\,ds$$

As used in an earlier chapter, $ds=\sqrt{(\frac{dx}{dt})^2+(\frac{dy}{dt})^2}$

Therefore, the line integral may be found by using:

$$\int_C f(x,y)\,ds=\int_a^b f(x(t),y(t)) \sqrt{\left(\frac{dx}{dt}\right)^2+\left(\frac{dy}{dt}\right)^2}\,dt$$

Much like any other integral, when applied to a piecewise function, the integral may be broken up:

$$\int_C f(x,y)\,ds=\int_{C_1} f(x,y)\,ds+\int_{C_2} f(x,y)\,ds+\int_{C_3} f(x,y)\,ds+\dots \int_{C_n} f(x,y)\,ds+$$

\end{document}
