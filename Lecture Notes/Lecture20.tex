%%%%%%%%%%%%%%%%%%%%%%%%%%%%%%%%%%%%%%%%%%%%%%%%%%%%%%%%%%%%%%%%%%%%%%%%%%%%%%%%%%%%%%%%%%%%%%%%%%%%%%%%%%%%%%%%%%%%%%%%%%%%%%%%%%%%%%%%%%%%%%%%%%%%%%%%%%%%%%%%%%%%%%%%%%%%%%%%%%%%%%%%%%%%
% Written By Michael Brodskiy
% Class: Analytic Geometry & Calculus III (Math-292)
% Professor: V. Cherkassky
%%%%%%%%%%%%%%%%%%%%%%%%%%%%%%%%%%%%%%%%%%%%%%%%%%%%%%%%%%%%%%%%%%%%%%%%%%%%%%%%%%%%%%%%%%%%%%%%%%%%%%%%%%%%%%%%%%%%%%%%%%%%%%%%%%%%%%%%%%%%%%%%%%%%%%%%%%%%%%%%%%%%%%%%%%%%%%%%%%%%%%%%%%%%

\documentclass[12pt]{article} 
\usepackage{alphalph}
\usepackage[utf8]{inputenc}
\usepackage[russian,english]{babel}
\usepackage{titling}
\usepackage{amsmath}
\usepackage{graphicx}
\usepackage{enumitem}
\usepackage{amssymb}
\usepackage[super]{nth}
\usepackage{everysel}
\usepackage{ragged2e}
\usepackage{geometry}
\usepackage{fancyhdr}
\geometry{top=1.0in,bottom=1.0in,left=1.0in,right=1.0in}
\newcommand{\subtitle}[1]{%
  \posttitle{%
    \par\end{center}
    \begin{center}\large#1\end{center}
    \vskip0.5em}%

}
\usepackage{hyperref}
\hypersetup{
colorlinks=true,
linkcolor=blue,
filecolor=magenta,      
urlcolor=blue,
citecolor=blue,
}

\urlstyle{same}


\title{Lecture XX Notes}
\date{\today}
\author{Michael Brodskiy\\ \small Professor: V. Cherkassky}

% Mathematical Operations:

% Sum: $$\sum_{n=a}^{b} f(x) $$
% Integral: $$\int_{lower}^{upper} f(x) dx$$
% Limit: $$\lim_{x\to\infty} f(x)$$

\begin{document}

\maketitle

\section{Line Integrals (Continued) $-$ 16.2}

A line integral defined by the function $f(x,y,z)$ may be integrated by using the formula:

$$\int_C f(x,y,z)\,ds=\int_C f(x(t),y(t),z(t))\sqrt{\left(\frac{dx}{dt}\right)^2+\left(\frac{dy}{dt}\right)^2+\left(\frac{dz}{dt}\right)^2}\,dt$$

Given a vector function, $\overrightarrow{r}(t)$, the above formula may be simplified to yield:

$$\int_C f(x,y,z)\,ds=\int_C f(\overrightarrow{r}(t))|\overrightarrow{r}'(t)|\,dt$$

Line integrals may also appear with respect to $dx$ and $dy$. This may be simplified by finding $dx$ and $dy$ in terms of $dt$:

$$\int_C f(x,y)\,dx = \int_a^b f(x(t),y(t))x'(t)\,dt$$
$$\int_C f(x,y)\,dy = \int_a^b f(x(t),y(t))y'(t)\,dt$$

When line integrals appear in such a form it is abbreviate them:

$$\int_C P(x,y)\,dx+\int_C Q(x,y)\,dy=\int_C P(x,y)\,dx+Q(x,y),dy$$

In physics, the formula for work was defined as $W=Fd\cos(\theta)$

This may be found by integrating:

$$W=\int_a^b \overrightarrow{F}(\overrightarrow{r}(t))\cdot\frac{\overrightarrow{r}'(t)}{|\overrightarrow{r}'(t)|}\,ds$$
$$W=\int_a^b \overrightarrow{F}(\overrightarrow{r}(t))\cdot\frac{\overrightarrow{r}'(t)}{|\overrightarrow{r}'(t)|}|\overrightarrow{r}'(t)|\,dt$$
$$W=\int_a^b \overrightarrow{F}(\overrightarrow{r}(t))\cdot\overrightarrow{r}'(t)\,dt$$
$$W=\int_a^b \overrightarrow{F}(\overrightarrow{r}(t))\cdot\overrightarrow{r}'(t)\,dt$$
$$W=\int_a^b \overrightarrow{F}\cdot\overrightarrow{r}\,dt$$

This yields the work done on a particle by a field.

\end{document}

