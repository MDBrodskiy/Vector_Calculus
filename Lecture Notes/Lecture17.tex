%%%%%%%%%%%%%%%%%%%%%%%%%%%%%%%%%%%%%%%%%%%%%%%%%%%%%%%%%%%%%%%%%%%%%%%%%%%%%%%%%%%%%%%%%%%%%%%%%%%%%%%%%%%%%%%%%%%%%%%%%%%%%%%%%%%%%%%%%%%%%%%%%%%%%%%%%%%%%%%%%%%%%%%%%%%%%%%%%%%%%%%%%%%%
% Written By Michael Brodskiy
% Class: Analytic Geometry & Calculus III (Math-292)
% Professor: V. Cherkassky
%%%%%%%%%%%%%%%%%%%%%%%%%%%%%%%%%%%%%%%%%%%%%%%%%%%%%%%%%%%%%%%%%%%%%%%%%%%%%%%%%%%%%%%%%%%%%%%%%%%%%%%%%%%%%%%%%%%%%%%%%%%%%%%%%%%%%%%%%%%%%%%%%%%%%%%%%%%%%%%%%%%%%%%%%%%%%%%%%%%%%%%%%%%%

\documentclass[12pt]{article} 
\usepackage{alphalph}
\usepackage[utf8]{inputenc}
\usepackage[russian,english]{babel}
\usepackage{titling}
\usepackage{amsmath}
\usepackage{graphicx}
\usepackage{enumitem}
\usepackage{amssymb}
\usepackage[super]{nth}
\usepackage{everysel}
\usepackage{ragged2e}
\usepackage{geometry}
\usepackage{fancyhdr}
\geometry{top=1.0in,bottom=1.0in,left=1.0in,right=1.0in}
\newcommand{\subtitle}[1]{%
  \posttitle{%
    \par\end{center}
    \begin{center}\large#1\end{center}
    \vskip0.5em}%

}
\usepackage{hyperref}
\hypersetup{
colorlinks=true,
linkcolor=blue,
filecolor=magenta,      
urlcolor=blue,
citecolor=blue,
}

\urlstyle{same}


\title{Lecture XVII Notes}
\date{July 13, 2020}
\author{Michael Brodskiy\\ \small Professor: V. Cherkassky}

% Mathematical Operations:

% Sum: $$\sum_{n=a}^{b} f(x) $$
% Integral: $$\int_{lower}^{upper} f(x) dx$$
% Limit: $$\lim_{x\to\infty} f(x)$$

\begin{document}

\maketitle

\section{Mass and Density $-$ 15.4}

The mass may be expressed as 
$$m=\iint_D \rho(x,y)\,dA$$

Total charge is expressed as:
$$Q=\iint_D \sigma(x,y)\,dA$$

The (first) moments on the $x$ and $y$ axis, respectively, may be found by using:

$$M_x=\iint_D y\rho(x,y)\,dA\text{ and }M_y=\iint_D x\rho(x,y)\,dA$$

The center of mass, $(\overline{x},\overline{y})$ may be found by rearranging the formula:

$$m\overline{x}=M_y\text{ and } m\overline{y}=M_x$$
$$\overline{x}=\frac{M_y}{m}=\frac{\iint_D x\rho(x,y)\,dA}{\iint_D \rho(x,y)}\,dA$$
  $$\overline{y}=\frac{M_x}{m}=\frac{\iint_D y\rho(x,y)\,dA}{\iint_D \rho(x,y)}\,dA$$

  The moment of inertia (or the second moment) is $mr^2$, where $r$ is the distance from a particle to the axis. About the $x$ axis, the moment of inertia is equal to:

  $$I_x = \iint_D y^2\rho(x,y)\,dA$$

  Across the $y$ axis the moment of inertia is equal to:

  $$I_y=\iint_D x^2\rho(x,y)\,dA$$

  In addition to this, the polar moment of inertia (about the origin) is equal to:

  $$I_0\iint_D (x^2+y^2)\rho(x,y)\,dA$$

  This means that $I_0=I_x+I_y$

  \section{Surface Area $-$ 15.5}

  The surface area, $A(S)$ is defined by:

  $$A(S)=\lim_{(m,n)\to\infty}\sum_{i=1}^m\sum_{j=1}^n \Delta T_ij$$

  $T_ij$ is defined by $|a\text{ x }b|$, where:

  $$a=\Delta x\bold{\hat{i}} + f_x(x_i,y_i)\Delta x \bold{\hat{k}}$$
  $$b=\Delta y\bold{\hat{j}} + f_y(x_i,y_j)\Delta y \bold{\hat{k}}$$

  Therefore, as an iterated integral, the surface area may be expressed as:

  $$A(S)=\iint_D \sqrt{1+\left(\frac{\partial z}{\partial x}\right)^2+\left(\frac{\partial z}{\partial y}\right)^2}\,dA$$


\end{document}
