%%%%%%%%%%%%%%%%%%%%%%%%%%%%%%%%%%%%%%%%%%%%%%%%%%%%%%%%%%%%%%%%%%%%%%%%%%%%%%%%%%%%%%%%%%%%%%%%%%%%%%%%%%%%%%%%%%%%%%%%%%%%%%%%%%%%%%%%%%%%%%%%%%%%%%%%%%%%%%%%%%%%%%%%%%%%%%%%%%%%%%%%%%%%
% Written By Michael Brodskiy
% Class: Analytic Geometry & Calculus III (Math-292)
% Professor: V. Cherkassky
%%%%%%%%%%%%%%%%%%%%%%%%%%%%%%%%%%%%%%%%%%%%%%%%%%%%%%%%%%%%%%%%%%%%%%%%%%%%%%%%%%%%%%%%%%%%%%%%%%%%%%%%%%%%%%%%%%%%%%%%%%%%%%%%%%%%%%%%%%%%%%%%%%%%%%%%%%%%%%%%%%%%%%%%%%%%%%%%%%%%%%%%%%%%

\documentclass[12pt]{article} 
\usepackage{alphalph}
\usepackage[utf8]{inputenc}
\usepackage[russian,english]{babel}
\usepackage{titling}
\usepackage{amsmath}
\usepackage{graphicx}
\usepackage{enumitem}
\usepackage{amssymb}
\usepackage[super]{nth}
\usepackage{everysel}
\usepackage{ragged2e}
\usepackage{geometry}
\usepackage{fancyhdr}
\geometry{top=1.0in,bottom=1.0in,left=1.0in,right=1.0in}
\newcommand{\subtitle}[1]{%
  \posttitle{%
    \par\end{center}
    \begin{center}\large#1\end{center}
    \vskip0.5em}%

}
\usepackage{hyperref}
\hypersetup{
colorlinks=true,
linkcolor=blue,
filecolor=magenta,      
urlcolor=blue,
citecolor=blue,
}

\urlstyle{same}


\title{Lecture XV Notes}
\date{\today}
\author{Michael Brodskiy\\ \small Professor: V. Cherkassky}

% Mathematical Operations:

% Sum: $$\sum_{n=a}^{b} f(x) $$
% Integral: $$\int_{lower}^{upper} f(x) dx$$
% Limit: $$\lim_{x\to\infty} f(x)$$

\begin{document}

\maketitle

\section{Double Integrals over General Regions $-$ 15.2}

First of all, $f$ must be continuous on a region $D$:

$$\{(x,y)|a\leq x\leq b, g_1(x)\leq y\leq g_2(x)\}$$

Such regions are categorized as Type I, and may be found using the following:

$$\iint_D f(x,y)\,dA=\int_a^b\int_{g_1(x)}^{g_2(x)} f(x,y)\,dy\,dx$$

A Type II equation must meet the same criteria as Type I, but with functions of $x$:

$$\{(x,y)|c\leq y\leq d, h_1(x)\leq x\leq h_2(x)\}$$

Such regions may be solved using the following:

$$\iint_D f(x,y)\,dA=\int_c^d\int_{h_1(y)}^{h_2(y)} f(x,y)\,dx\,dy$$

If the region, $D$, is neither Type I nor II, then the regions may be broken into two parts, $D_1$ and $D_2$:

$$\iint_D f(x,y)\,dA=\iint_{D_1} f(x,y)\,dA+\iint_{D_2} f(x,y)\,dA$$

Furthermore, if the function being integrated is a constant, $c$, then:

$$\iint_D c\,dA=cA(D)$$

Where $A(D)$ represents the area of the region $D$

This property may be used if $m\leq f(x,y)\leq M$ to achieve:

$$mA(D) \leq \iint_D f(x,y)\,dA \leq MA(D)$$


\end{document}
