%%%%%%%%%%%%%%%%%%%%%%%%%%%%%%%%%%%%%%%%%%%%%%%%%%%%%%%%%%%%%%%%%%%%%%%%%%%%%%%%%%%%%%%%%%%%%%%%%%%%%%%%%%%%%%%%%%%%%%%%%%%%%%%%%%%%%%%%%%%%%%%%%%%%%%%%%%%%%%%%%%%%%%%%%%%%%%%%%%%%%%%%%%%%
% Written By Michael Brodskiy
% Class: Analytic Geometry & Calculus III (Math-292)
% Professor: V. Cherkassky
%%%%%%%%%%%%%%%%%%%%%%%%%%%%%%%%%%%%%%%%%%%%%%%%%%%%%%%%%%%%%%%%%%%%%%%%%%%%%%%%%%%%%%%%%%%%%%%%%%%%%%%%%%%%%%%%%%%%%%%%%%%%%%%%%%%%%%%%%%%%%%%%%%%%%%%%%%%%%%%%%%%%%%%%%%%%%%%%%%%%%%%%%%%%

\documentclass[12pt]{article} 
\usepackage{alphalph}
\usepackage[utf8]{inputenc}
\usepackage[russian,english]{babel}
\usepackage{titling}
\usepackage{amsmath}
\usepackage{graphicx}
\usepackage{enumitem}
\usepackage{amssymb}
\usepackage[super]{nth}
\usepackage{everysel}
\usepackage{ragged2e}
\usepackage{geometry}
\usepackage{fancyhdr}
\geometry{top=1.0in,bottom=1.0in,left=1.0in,right=1.0in}
\newcommand{\subtitle}[1]{%
  \posttitle{%
    \par\end{center}
    \begin{center}\large#1\end{center}
    \vskip0.5em}%

}
\usepackage{hyperref}
\hypersetup{
colorlinks=true,
linkcolor=blue,
filecolor=magenta,      
urlcolor=blue,
citecolor=blue,
}

\urlstyle{same}


\title{Lecture V Notes}
\date{\today}
\author{Michael Brodskiy\\ \small Professor: V. Cherkassky}

% Mathematical Operations:

% Sum: $$\sum_{n=a}^{b} f(x) $$
% Integral: $$\int_{lower}^{upper} f(x) dx$$
% Limit: $$\lim_{x\to\infty} f(x)$$

\begin{document}

\maketitle

\section{Vectors in Space $-$ 13.1}

A vector function: $\overrightarrow{r}(t) = \langle f(t), g(t), h(t) \rangle = f(t)\hat{\bold{i}} + g(t)\hat{\bold{j}} + h(t)\hat{\bold{k}}$\\

Example: $$\overrightarrow{r}(t)=\langle t^2, ln(2-t), \sqrt{t} \rangle$$
The domains of each component function must be considered to find the domain of the vector function, $0\leq x < 2$\\

For a vector function: $$\lim_{t\to a} \overrightarrow{r}(t)=\langle \lim_{t\to a} f(t), \lim_{t\to a} g(t), \lim_{t\to a} h(t) \rangle$$\\

Example: Find the limit
$$\lim_{t\to 0} (e^{-3t}\hat{\bold{i}} + \frac{t^2}{\sin^2{t}}\hat{\bold{j}} + \cos(2t)\hat{\bold{k}})=\hat{\bold{i}}+\hat{\bold{j}}+\hat{\bold{k}}$$

\section{Derivatives and Integrals of Vector Functions $-$ 13.2}

$$\overrightarrow{r}'(t)=\lim_{h\to 0} \frac{\overrightarrow{r}(t+h)-\overrightarrow{r}(t)}{h}$$

$\overrightarrow{r}'(t)$ is a vector tangent to $\overrightarrow{r}(t)$, where $\overrightarrow{r}'(t)=\langle f'(t), g'(t), h'(t) \rangle$

The unit tangent vector, $\overrightarrow{T}(t)$, may found using the formula: $\overrightarrow{T}(t)=\frac{\overrightarrow{r}'(t)}{|\overrightarrow{r}'(t)|}$

The integral of the vector function $\overrightarrow{r}(t)$ is: $$\int_{a}^{b} \overrightarrow{r}(t) dt = \left(\int_{a}^{b} f(t) dt\right)\hat{\bold{i}} + \left(\int_{a}^{b} g(t) dt\right)\hat{\bold{j}} + \left(\int_{a}^{b} h(t) dt\right)\hat{\bold{k}}$$

\end{document}
