%%%%%%%%%%%%%%%%%%%%%%%%%%%%%%%%%%%%%%%%%%%%%%%%%%%%%%%%%%%%%%%%%%%%%%%%%%%%%%%%%%%%%%%%%%%%%%%%%%%%%%%%%%%%%%%%%%%%%%%%%%%%%%%%%%%%%%%%%%%%%%%%%%%%%%%%%%%%%%%%%%%%%%%%%%%%%%%%%%%%%%%%%%%%
% Written By Michael Brodskiy
% Class: Analytic Geometry & Calculus III (Math-292)
% Professor: V. Cherkassky
%%%%%%%%%%%%%%%%%%%%%%%%%%%%%%%%%%%%%%%%%%%%%%%%%%%%%%%%%%%%%%%%%%%%%%%%%%%%%%%%%%%%%%%%%%%%%%%%%%%%%%%%%%%%%%%%%%%%%%%%%%%%%%%%%%%%%%%%%%%%%%%%%%%%%%%%%%%%%%%%%%%%%%%%%%%%%%%%%%%%%%%%%%%%

\documentclass[12pt]{article} 
\usepackage{alphalph}
\usepackage[utf8]{inputenc}
\usepackage[russian,english]{babel}
\usepackage{titling}
\usepackage{amsmath}
\usepackage{graphicx}
\usepackage{enumitem}
\usepackage{amssymb}
\usepackage[super]{nth}
\usepackage{everysel}
\usepackage{ragged2e}
\usepackage{geometry}
\usepackage{fancyhdr}
\geometry{top=1.0in,bottom=1.0in,left=1.0in,right=1.0in}
\newcommand{\subtitle}[1]{%
  \posttitle{%
    \par\end{center}
    \begin{center}\large#1\end{center}
    \vskip0.5em}%

}
\usepackage{hyperref}
\hypersetup{
colorlinks=true,
linkcolor=blue,
filecolor=magenta,      
urlcolor=blue,
citecolor=blue,
}

\urlstyle{same}


\title{Lecture IV Notes}
\date{\today}
\author{Michael Brodskiy\\ \small Professor: V. Cherkassky}


\begin{document}

\maketitle

\subsection{The Distance From a Point to a Plane}

The distance from any point to a plane may be found using the formula $\frac{|ax+by+cz+d|}{\sqrt{a^2+b^2+c^2}}$, where $a$-$d$ are the coefficients of the equation of the plane, and $x$-$z$ are the coordinates of the point.

\subsection{The Distance From a Plane to a Plane}

Find one point on either plane, and repeat the process to find a distance from a point to a plane.

\section{Cylinders and Quadric Surfaces}

The equation of a quadric surface is given by the equation: $Ax^2+By^2+Cz^2+Dxy+Eyz+Fxz+Gx+Hy+Iz+J$, where $A$-$J$ are constants.\\

There are six different figures that should be known:

\begin{tabular}{|p{.45\textwidth}||p{.45\textwidth}|}

\hline
  Figure & Equation \\
\hline
Ellipsoid: A Figure in Which All Traces are Ellipses & $\frac{x^2}{a^2}+\frac{y^2}{b^2}+\frac{z^2}{c^2}=1$\\
\hline
  Cone: A Figure in Which Horizontal Traces are Ellipses and Vertical Traces in $x$ and $y$ are Hyperbolas & $\frac{x^2}{a^2}+\frac{y^2}{b^2}=\frac{z^2}{c^2}$\\
\hline
Elliptic Paraboloid: Horizontal Traces are Ellipses and Vertical Traces are Parabolas & $\frac{x^2}{a^2}+\frac{y^2}{b^2}=\frac{z}{c}$\\ 
\hline
Hyperboloid of One Sheet: Horizontal Traces are Ellipses and Vertical Traces are Hyperbolas & $\frac{x^2}{a^2}+\frac{y^2}{b^2}-\frac{z^2}{c^2}=1$\\ 
\hline
Hyperbolic Paraboloid: Horizontal Traces are Hyperbolas and Vertical Traces are Parabolas & $\frac{x^2}{a^2}-\frac{y^2}{b^2}=\frac{z}{c}$\\
\hline
Hyperboloid of Two Sheets: Horizontal Traces are Ellipses in $z$ and Vertical Traces are Hyperbolas & $-\frac{x^2}{a^2}-\frac{y^2}{b^2}+\frac{z^2}{c^2}=1$\\
\hline

\end{tabular}

\end{document}
