%%%%%%%%%%%%%%%%%%%%%%%%%%%%%%%%%%%%%%%%%%%%%%%%%%%%%%%%%%%%%%%%%%%%%%%%%%%%%%%%%%%%%%%%%%%%%%%%%%%%%%%%%%%%%%%%%%%%%%%%%%%%%%%%%%%%%%%%%%%%%%%%%%%%%%%%%%%%%%%%%%%%%%%%%%%%%%%%%%%%%%%%%%%%
% Written By Michael Brodskiy
% Class: Analytic Geometry & Calculus III (Math-292)
% Professor: V. Cherkassky
%%%%%%%%%%%%%%%%%%%%%%%%%%%%%%%%%%%%%%%%%%%%%%%%%%%%%%%%%%%%%%%%%%%%%%%%%%%%%%%%%%%%%%%%%%%%%%%%%%%%%%%%%%%%%%%%%%%%%%%%%%%%%%%%%%%%%%%%%%%%%%%%%%%%%%%%%%%%%%%%%%%%%%%%%%%%%%%%%%%%%%%%%%%%

\documentclass[12pt]{article} 
\usepackage{alphalph}
\usepackage[utf8]{inputenc}
\usepackage[russian,english]{babel}
\usepackage{titling}
\usepackage{amsmath}
\usepackage{graphicx}
\usepackage{enumitem}
\usepackage{amssymb}
\usepackage{gensymb}
\usepackage[super]{nth}
\usepackage{everysel}
\usepackage{ragged2e}
\usepackage{geometry}
\usepackage{fancyhdr}
\geometry{top=1.0in,bottom=1.0in,left=1.0in,right=1.0in}
\newcommand{\subtitle}[1]{%
  \posttitle{%
    \par\end{center}
    \begin{center}\large#1\end{center}
    \vskip0.5em}%

}
\usepackage{hyperref}
\hypersetup{
colorlinks=true,
linkcolor=blue,
filecolor=magenta,      
urlcolor=blue,
citecolor=blue,
}

\urlstyle{same}


\title{Final Project $-$ Chapter 15}
\date{July 11, 2020}
\author{Michael Brodskiy\\ \small Professor: V. Cherkassky}


\begin{document}

\maketitle

\begin{center}

Page 1061

\end{center}

\begin{enumerate}

\setcounter{enumi}{28}

  \item \#2 \begin{enumerate}

      \item A rectangular $D$ is defined with $x$ and $y$ being confined by numbers only. In a general region, $D$ has one of two variables confined by functions: $h_1(y)\leq x\leq h_2(y)$, or $g_1(x)\leq y\leq g_2(x)$

      \item A Type I region is created when the $y$ variable is bounded by functions: $g_1(x)\leq y\leq g_2(x)$. These may be evaluated by finding: 
        $$\iint_D f(x,y)\,dA = \int_a^b\int_{g_1(x)}^{g_2(x)} f(x,y)\,dy\,dx$$

      \item A Type II region is created when the $x$ variable is bounded by functions: $h_1(y)\leq x\leq h_2(y)$. These may be evaluated by finding: 
        $$\iint_D f(x,y)\,dA = \int_c^d\int_{h_1(y)}^{h_2(y)} f(x,y)\,dx\,dy$$

      \item If necessary, a double integral may be split into two regions: 
        $$\iint_D f(x,y)\,dA=\iint_{D_1}f(x,y)\,dA+\iint_{D_2}f(x,y)dA$$
In addition to this, if a function of $x$ is multiplied by a function of $y$, the two functions may be integrated separately:
$$\iint_Dg(x)h(y)\,dA=\int_a^bg(x)\,dx\int_c^dh(y)\,dy$$
As with single integrals, a constant may be taken out of the integrand:
$$\iint_D cf(x,y)\,dA=c\iint_Df(x,y)\,dA$$
Also, added functions may be separated:
$$\iint_D [f(x,y)+g(x,y)]\,dA=\iint_D f(x,y)\,dA+\iint_D g(x,y)\,dA$$
If $1$ is integrated, the area of $D$ will be found:
$$\iint_D 1\,dA=A(D)$$
If there are two constants, $m$ and $M$, and $m\leq f(x,y)\leq M$ for all $(x,y)$ in $D$, then:
$$mA(D)\leq \iint_D f(x,y)\,dA\leq MA(D)$$

    \end{enumerate}

  \item \#6 \begin{enumerate}

      \item To find the surface area, one would simply use the surface area formula:
        $$A(S)=\iint_D\sqrt{f_x(x,y)^2+f_y(x,y)^2 + 1}\,dA$$

    \end{enumerate}

    \begin{center}

Page 1062

    \end{center}

    \item \#3

    \begin{enumerate}

      \item The work is as follows:
        $$\int_1^2\int_0^2 (y+2xe^y)\,dx\,dy$$
        $$\int_1^2(2y+4e^y)\,dy$$
        $$y^2+4e^y\Big|_1^2\Longrightarrow 4e^2-4e+3$$
    \end{enumerate}


    \item \#6 \begin{enumerate}

        \item The work is as follows:
          $$\int_0^1\int_x^{e^x}3xy^2\,dy\,dx$$
          $$\int_0^1 xe^{3x}-x^4\,dx$$
          $$\frac{1}{3}xe^{3x}-\frac{1}{9}e^{3x}-\frac{1}{5}x^5\Big|_0^1 \Longrightarrow \frac{2}{9}e^3-\frac{4}{45}$$

\end{enumerate}

    \item \#9 \begin{enumerate}

        \item The work is as follows:
          $$D=\{(r,\theta)|2\leq r\leq 4, 0\leq \theta\leq \pi\}$$
          $$\int_0^{\pi}\int_2^4f(r\cos\theta,r\sin\theta)r\,dr\,d\theta$$


    \end{enumerate}

  \item \#19 \begin{enumerate}

      \item The work is as follows:
        $$\int_0^1 1\,dx\int_x^1 \cos(y^2)\,dy$$
        $$\frac{1}{2y}\sin(y^2)$$
        $$\frac{1}{2}\sin(1)$$

        \end{enumerate}

    \item \#23 \begin{enumerate}

        \item The work is as follows:
          $$\int_0^1\int_0^{\sqrt{x}} \frac{y}{1+x^2}\,dy\,dx$$
          $$\frac{1}{2}\int_0^1 \frac{x}{1+x^2}\,dx$$
          $$\frac{1}{4}\int_1^2 \frac{1}{u}\,du$$
          $$\frac{1}{4}\ln(u)\Big|_1^2=\frac{1}{4}\ln(2)$$


      \end{enumerate}

    \item \#27 \begin{enumerate}

        \item This would be much easier in polar coordinates. Therefore, the angle between the line $y=\sqrt{3}x$ and the $x$ axis must be found. The angle turns out to be $\frac{\pi}{3}$, and the radius is 3. Then, the iterated integrals may be set up: 
          $$\int_0^{\frac{\pi}{3}}\int_0^3 r^4\,dr\,d\theta$$
          $$\int_0^{\frac{\pi}{3}} \frac{243}{5}\,d\theta$$
          $$\frac{243}{5}\theta\Big|_0^{\frac{\pi}{3}}=\frac{81\pi}{5}$$

      \end{enumerate}

    \item \#35 \begin{enumerate}

        \item The work is as follows:
          $$\int_0^2\int_1^4 x^2+4y^2\,dy\,dx$$
          $$\int_0^2 3x^2+84\,dx$$
          $$x^3+84x\Big|_0^2=176$$

      \end{enumerate}

    \item \#37 \begin{enumerate}

        \item The given points result in a pyramid. This pyramid has an isosceles base, with side lengths 2. This means the area of the base is equal to $2$. Furthermore, the height of the pyramid is $1$ using this in the formula for the volume of a pyramid results in: $\frac{1}{3}(2)(1)=\frac{2}{3}$  

      \end{enumerate}

	\item \#53 \begin{enumerate}
	
		\item The work is as follows:
        $$\int_{-1}^1\int_{x^2}^1\int_0^{1-y}\f(x,y,z)\,dx\,dy\,dz\Longrightarrow\int_0^{1}\int_{0}^{1-z}\int_{-\sqrt{y}}^{\sqrt{y}}f(x,y,z)\,dx\,dy\,dz$$
	
	\end{enumerate}
	
	\item \#55 \begin{enumerate}
	
		\item The work is as follows:
          $$J=\frac{1}{2}$$
          $$\int_2^4\int_{-2}^0\frac{u}{2v}\,du\,dv$$
          $$-\int_{2}^4 \frac{1}{v}\,dv=-\ln(2) $$

	\end{enumerate}
	
\end{enumerate}


\end{document}
