%%%%%%%%%%%%%%%%%%%%%%%%%%%%%%%%%%%%%%%%%%%%%%%%%%%%%%%%%%%%%%%%%%%%%%%%%%%%%%%%%%%%%%%%%%%%%%%%%%%%%%%%%%%%%%%%%%%%%%%%%%%%%%%%%%%%%%%%%%%%%%%%%%%%%%%%%%%%%%%%%%%%%%%%%%%%%%%%%%%%%%%%%%%%
% Written By Michael Brodskiy
% Class: Analytic Geometry & Calculus III (Math-292)
% Professor: V. Cherkassky
%%%%%%%%%%%%%%%%%%%%%%%%%%%%%%%%%%%%%%%%%%%%%%%%%%%%%%%%%%%%%%%%%%%%%%%%%%%%%%%%%%%%%%%%%%%%%%%%%%%%%%%%%%%%%%%%%%%%%%%%%%%%%%%%%%%%%%%%%%%%%%%%%%%%%%%%%%%%%%%%%%%%%%%%%%%%%%%%%%%%%%%%%%%%

\documentclass[12pt]{article} 
\usepackage{alphalph}
\usepackage[utf8]{inputenc}
\usepackage[russian,english]{babel}
\usepackage{titling}
\usepackage{amsmath}
\usepackage{graphicx}
\usepackage{enumitem}
\usepackage{amssymb}
\usepackage[super]{nth}
\usepackage{everysel}
\usepackage{ragged2e}
\usepackage{geometry}
\usepackage{fancyhdr}
\geometry{top=1.0in,bottom=1.0in,left=1.0in,right=1.0in}
\newcommand{\subtitle}[1]{%
  \posttitle{%
    \par\end{center}
    \begin{center}\large#1\end{center}
    \vskip0.5em}%

}
\usepackage{hyperref}
\hypersetup{
colorlinks=true,
linkcolor=blue,
filecolor=magenta,      
urlcolor=blue,
citecolor=blue,
}

\urlstyle{same}


\title{Final Project $-$ Chapter 12}
\date{\today}
\author{Michael Brodskiy\\ \small Professor: V. Cherkassky}


\begin{document}

\maketitle

\begin{center}

Page 841

\end{center}

\begin{enumerate}

  \item \#15 \begin{enumerate}

      \item Two vectors are parallel if they are scalar multiples of each other. For example, if $\overrightarrow{a}=<a_1,a_2,a_3>$, a parallel vector would equal $c\overrightarrow{a}$, where c is a scalar. Therefore, if a vector is equal to another vector multiplied by a scalar, they are parallel. Also, two vectors are parallel if their cross product is equal to zero.

     \item Two vectors are orthogonal (perpendicular) simply if their dot product is equal to zero.

     \item It can be determined whether two planes are parallel by using the vectors normal to these two planes. If the two normal vectors are parallel (which can be determined using the method from (a)), then the planes are parallel as well. 

    \end{enumerate}

  \item \#16 \begin{enumerate}

      \item One can take the three points, $\overrightarrow{P}, \overrightarrow{Q}, \text{ and } \overrightarrow{R}$, and form two vectors. Then, if the two vectors are scalar multiples of each other, then these points are collinear.

      \item From four points, one can find three vectors, say $\overrightarrow{PQ}$, $\overrightarrow{PR}$, and $\overrightarrow{PS}$. By crossing, $\overrightarrow{PQ}$ x $\overrightarrow{PR}$, one can find a vector orthogonal, say, $\overrightarrow{n}$ to both $\overrightarrow{PQ}$ and $\overrightarrow{PR}$. Then, if the crossed vector, $\overrightarrow{n}$, is orthogonal to the third vector, $\overrightarrow{PS}$, (which can be determined if the dot product, $\overrightarrow{n}\cdot\overrightarrow{PS}=0$) the four points lie on one plane. 

    \end{enumerate}

  \item \#17 \begin{enumerate}

      \item First, one must choose two points on the line, points $A$ and $B$. Then, two vectors must be formed, $\overrightarrow{AB} \text{ and } \overrightarrow{AP} \text{ or } \overrightarrow{BP}$, where $P$ is the point. Then, the distance can be found using the formula: $\frac{|\overrightarrow{AB}\text{ x }(\overrightarrow{AP} \text{ or } \overrightarrow{BP})|}{|\overrightarrow{AB}|}$  

      \item The distance from a point to a plane may be found using the formula: $\frac{|ax+by+cz+d|}{\sqrt{a^2+b^2+c^2}}$, where $a$-$d$ are the coefficients of the standard form of the plane, and $x$-$z$ are respective coordinates of the point.

      \item By determining one point on a line, say $L_1$, and two points on line $L_2$, the same process from part (a) may be used.

    \end{enumerate}

    \begin{center}

Page 843

    \end{center}

  \item $\overrightarrow{v}\Longrightarrow\langle1-4,1-(-1),5-2\rangle=\langle-3, 2, 3\rangle$ \\ $r=r_o+tv \Longrightarrow x=4-3t, y=-1+2t, z=2+3t \Longrightarrow r = \langle 4-3t, -1+2t, 2+3t\rangle$

\end{enumerate}


\end{document}
