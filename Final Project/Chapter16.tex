%%%%%%%%%%%%%%%%%%%%%%%%%%%%%%%%%%%%%%%%%%%%%%%%%%%%%%%%%%%%%%%%%%%%%%%%%%%%%%%%%%%%%%%%%%%%%%%%%%%%%%%%%%%%%%%%%%%%%%%%%%%%%%%%%%%%%%%%%%%%%%%%%%%%%%%%%%%%%%%%%%%%%%%%%%%%%%%%%%%%%%%%%%%%
% Written By Michael Brodskiy
% Class: Analytic Geometry & Calculus III (Math-292)
% Professor: V. Cherkassky
%%%%%%%%%%%%%%%%%%%%%%%%%%%%%%%%%%%%%%%%%%%%%%%%%%%%%%%%%%%%%%%%%%%%%%%%%%%%%%%%%%%%%%%%%%%%%%%%%%%%%%%%%%%%%%%%%%%%%%%%%%%%%%%%%%%%%%%%%%%%%%%%%%%%%%%%%%%%%%%%%%%%%%%%%%%%%%%%%%%%%%%%%%%%

\documentclass[12pt]{article} 
\usepackage{alphalph}
\usepackage[utf8]{inputenc}
\usepackage[russian,english]{babel}
\usepackage{titling}
\usepackage{amsmath}
\usepackage{graphicx}
\usepackage{enumitem}
\usepackage{amssymb}
\usepackage{gensymb}
\usepackage[super]{nth}
\usepackage{everysel}
\usepackage{ragged2e}
\usepackage{geometry}
\usepackage{fancyhdr}
\geometry{top=1.0in,bottom=1.0in,left=1.0in,right=1.0in}
\newcommand{\subtitle}[1]{%
  \posttitle{%
    \par\end{center}
    \begin{center}\large#1\end{center}
    \vskip0.5em}%

}
\usepackage{hyperref}
\hypersetup{
colorlinks=true,
linkcolor=blue,
filecolor=magenta,      
urlcolor=blue,
citecolor=blue,
}

\urlstyle{same}


\title{Final Project $-$ Chapter 16}
\date{July 18, 2020}
\author{Michael Brodskiy\\ \small Professor: V. Cherkassky}


\begin{document}

\maketitle

\begin{center}

Page 1148

\end{center}

\begin{enumerate}

\setcounter{enumi}{40}

  \item \#5 \begin{enumerate}

      \item  

    \end{enumerate}

  \item \#6 \begin{enumerate}

      \item 

    \end{enumerate}


    \item \#7

    \begin{enumerate}

      \item 
      
    \end{enumerate}

\begin{center}

Page 1149

\end{center}

    \item \#1 \begin{enumerate}

        \item The work is negative because it is done in a direction opposing the vector field.

\end{enumerate}

    \item \#3 \begin{enumerate}

        \item 

    \end{enumerate}

  \item \#5 \begin{enumerate}

			\item

        \end{enumerate}

    \item \#7 \begin{enumerate}

				\item


      \end{enumerate}

    \item \#11 \begin{enumerate}

        \item 

      \end{enumerate}

    \item \#13 \begin{enumerate}

        \item 

      \end{enumerate}

    \item \#15 \begin{enumerate}

        \item  

      \end{enumerate}

	
\end{enumerate}


\end{document}
