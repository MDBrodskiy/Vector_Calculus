%%%%%%%%%%%%%%%%%%%%%%%%%%%%%%%%%%%%%%%%%%%%%%%%%%%%%%%%%%%%%%%%%%%%%%%%%%%%%%%%%%%%%%%%%%%%%%%%%%%%%%%%%%%%%%%%%%%%%%%%%%%%%%%%%%%%%%%%%%%%%%%%%%%%%%%%%%%%%%%%%%%%%%%%%%%%%%%%%%%%%%%%%%%%
% Written By Michael Brodskiy
% Class: Analytic Geometry & Calculus III (Math-292)
% Professor: V. Cherkassky
%%%%%%%%%%%%%%%%%%%%%%%%%%%%%%%%%%%%%%%%%%%%%%%%%%%%%%%%%%%%%%%%%%%%%%%%%%%%%%%%%%%%%%%%%%%%%%%%%%%%%%%%%%%%%%%%%%%%%%%%%%%%%%%%%%%%%%%%%%%%%%%%%%%%%%%%%%%%%%%%%%%%%%%%%%%%%%%%%%%%%%%%%%%%

\documentclass[12pt]{article} 
\usepackage{alphalph}
\usepackage[utf8]{inputenc}
\usepackage[russian,english]{babel}
\usepackage{titling}
\usepackage{amsmath}
\usepackage{graphicx}
\usepackage{enumitem}
\usepackage{amssymb}
\usepackage{gensymb}
\usepackage[super]{nth}
\usepackage{everysel}
\usepackage{ragged2e}
\usepackage{geometry}
\usepackage{fancyhdr}
\geometry{top=1.0in,bottom=1.0in,left=1.0in,right=1.0in}
\newcommand{\subtitle}[1]{%
  \posttitle{%
    \par\end{center}
    \begin{center}\large#1\end{center}
    \vskip0.5em}%

}
\usepackage{hyperref}
\hypersetup{
colorlinks=true,
linkcolor=blue,
filecolor=magenta,      
urlcolor=blue,
citecolor=blue,
}

\urlstyle{same}


\title{Final Project $-$ Chapter 14}
\date{July 4, 2020}
\author{Michael Brodskiy\\ \small Professor: V. Cherkassky}


\begin{document}

\maketitle

\begin{center}

Page 981

\end{center}

\begin{enumerate}

\setcounter{enumi}{18}

  \item \#8 \begin{enumerate}

      \item The linearization of $f$ is a function which results in a plane tangent to the surface of $f(x,y)$ at the given point. 

      \item The formula for the approximation may be expressed as:
        $$f(x,y)\approx f(a,b)+\frac{\partial f}{\partial x}(x-a)+\frac{\partial f}{\partial y}(y-b)$$

      \item In geometric terms, the linear approximation may be used to approximate values near $(a,b)$

    \end{enumerate}

  \item \#13 \begin{enumerate}

      \item The directional derivative of $f$ (towards the direction of $\overrightarrow{\bold{u}}$) may be expressed as:
      $$D{_\overrightarrow{\bold{u}}}=\lim_{h\to0}\frac{f(x_o+ha,y_o+hb)-f(x_o,y_o)}{h}$$
This represents the rate of change (slope) of $f$ in the direction of $\overrightarrow{\bold{u}}$
In geometric terms, this represents the slope of the plane tangent to $f$ at a certain point.

      \item $D_{\overrightarrow{\bold{u}}}=f_x(x_o,y_o)a+f_y(x_o,y_o)b$ 

    \end{enumerate}

    \begin{center}

Page 982

    \end{center}

    \item \#9

    \begin{enumerate}

      \item $\lim_{(x,y)\to(1,1)}\frac{2xy}{x^2+2y^2}$ \\
        Test (Direct Plug-in): $\lim_{(x,y)\to(1,1)}\frac{2xy}{x^2+2y^2}=\frac{2}{3}$\\
        Test (along $y=mx$): $\lim_{(x,y)\to(1,1)}\frac{2mx^2}{(2m^2+1)x^2}$ if $m=1\Longrightarrow \frac{2}{3}$\\
        Test (along $y=x^2$): $\lim_{(x,y)\to(1,1)}\frac{2x^3}{x^2+2x^4}=\frac{2}{3}$
        $$\therefore\lim_{(x,y)\to(1,1)}\frac{2xy}{x^2+2y^2}=\frac{2}{3}$$

    \end{enumerate}

    \begin{center}

Page 983

    \end{center}


    \item \#13 \begin{enumerate}

        \item $\frac{\partial f}{\partial x} = 32xy(5y^3+2x^2y)^7, \frac{\partial f}{\partial y} = 8(15y^2+2x^2)(5y^3+2x^2y)^7$ 

\end{enumerate}

    \item \#17 \begin{enumerate}

        \item $\frac{\partial S}{\partial u}=\arctan(v\sqrt{w}), \frac{\partial S}{\partial v} = \frac{u\sqrt{w}}{1+v^2w}, \frac{\partial S}{\partial w} = \frac{uv}{2\sqrt{w}(1+v^2w)}$

    \end{enumerate}

  \item \#25 \begin{enumerate}

      \item $\frac{\partial z}{\partial x}=6x+2, \frac{\partial z}{\partial y}=-2y\Longrightarrow slope_x=8, slope_y=4$
        $$z=1+8(x-1)+4(y+2)\Longrightarrow z=8x+4y+1\text{ OR } z-4y-8x=1$$
      \item $\frac{x-1}{8}=\frac{y+2}{4}=1-z$

    \end{enumerate}

    \item \#31 \begin{enumerate}

        \item $z=\pm \sqrt{x^2+4y^2-4}$
          $$\pm\frac{\partial z}{\partial x}=\frac{x}{\sqrt{x^2+4y^2-4}}; \frac{\partial z}{\partial y} = \pm \frac{4y}{\sqrt{x^2+4y^2-4}}$$
          $$\frac{\partial z}{\partial x}, \frac{\partial z}{\partial y} = 2\text{ @ }x=\pm2, y=\pm\frac{1}{2}, z=\mp1$$
      \end{enumerate}

    \item \#32 \begin{enumerate}

        \item $\frac{\partial u}{\partial s}=\frac{e^{2t}}{1+se^{2t}}, \frac{\partial u}{\partial t}=\frac{2se^{2t}}{1+se^2t}$
          $$du=(\frac{e^{2t}}{1+se^{2t}})ds+(\frac{2se^{2t}}{1+se^2t})dt$$ 

      \end{enumerate}

    \item \#35 \begin{enumerate}

        \item First find the partials and derivatives:
          $$\frac{dx}{dp}=6p+1,\frac{dy}{dp}=e^p+pe^p,\frac{dz}{dp}=\sin(p)+p\cos(p)$$
          $$\frac{\partial u}{\partial x}=2xy^3, \frac{\partial u}{\partial y}=3x^2y^2,\frac{\partial u}{\partial z}=4z^3$$
          $$du=(6p+1)(2xy^3)+(e^p+pe^p)(3x^2y^2)+(\sin(p)+p\cos(p))(4z^3)$$

      \end{enumerate}

    \item \#39 \begin{enumerate}

        \item $$\frac{\partial z}{\partial x}=2xf(x^2-y^2), \frac{\partial z}{\partial y}=1-2yf(x^2-y^2)$$
          $$y(2xf(x^2-y^2))+x(1-2yf(x^2-y^2))\Longrightarrow x$$
          $$\therefore y\frac{\partial z}{\partial x}+x\frac{\partial z}{\partial y}=x$$

      \end{enumerate}



\end{enumerate}


\end{document}
