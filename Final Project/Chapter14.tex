%%%%%%%%%%%%%%%%%%%%%%%%%%%%%%%%%%%%%%%%%%%%%%%%%%%%%%%%%%%%%%%%%%%%%%%%%%%%%%%%%%%%%%%%%%%%%%%%%%%%%%%%%%%%%%%%%%%%%%%%%%%%%%%%%%%%%%%%%%%%%%%%%%%%%%%%%%%%%%%%%%%%%%%%%%%%%%%%%%%%%%%%%%%%
% Written By Michael Brodskiy
% Class: Analytic Geometry & Calculus III (Math-292)
% Professor: V. Cherkassky
%%%%%%%%%%%%%%%%%%%%%%%%%%%%%%%%%%%%%%%%%%%%%%%%%%%%%%%%%%%%%%%%%%%%%%%%%%%%%%%%%%%%%%%%%%%%%%%%%%%%%%%%%%%%%%%%%%%%%%%%%%%%%%%%%%%%%%%%%%%%%%%%%%%%%%%%%%%%%%%%%%%%%%%%%%%%%%%%%%%%%%%%%%%%

\documentclass[12pt]{article} 
\usepackage{alphalph}
\usepackage[utf8]{inputenc}
\usepackage[russian,english]{babel}
\usepackage{titling}
\usepackage{amsmath}
\usepackage{graphicx}
\usepackage{enumitem}
\usepackage{amssymb}
\usepackage{gensymb}
\usepackage[super]{nth}
\usepackage{everysel}
\usepackage{ragged2e}
\usepackage{geometry}
\usepackage{fancyhdr}
\geometry{top=1.0in,bottom=1.0in,left=1.0in,right=1.0in}
\newcommand{\subtitle}[1]{%
  \posttitle{%
    \par\end{center}
    \begin{center}\large#1\end{center}
    \vskip0.5em}%

}
\usepackage{hyperref}
\hypersetup{
colorlinks=true,
linkcolor=blue,
filecolor=magenta,      
urlcolor=blue,
citecolor=blue,
}

\urlstyle{same}


\title{Final Project $-$ Chapter 14}
\date{July 4, 2020}
\author{Michael Brodskiy\\ \small Professor: V. Cherkassky}


\begin{document}

\maketitle

\begin{center}

Page 981

\end{center}

\begin{enumerate}

\setcounter{enumi}{18}

  \item \#8 \begin{enumerate}

      \item The linearization of $f$ is a function which results in a plane tangent to the surface of $f(x,y)$ at the given point. 

      \item The formula for the approximation may be expressed as:
        $$f(x,y)\approx f(a,b)+\frac{\partial f}{\partial x}(x-a)+\frac{\partial f}{\partial y}(y-b)$$

      \item In geometric terms, the linear approximation may be used to approximate values near $(a,b)$

    \end{enumerate}

  \item \#13 \begin{enumerate}

      \item 

      \item 

    \end{enumerate}

    \begin{center}

Page 982

    \end{center}

    \item \#9

    \begin{enumerate}

      \item 

    \end{enumerate}

    \item \#13 \begin{enumerate}

        \item 

\end{enumerate}

    \item \#17 \begin{enumerate}

        \item 

    \end{enumerate}

  \item \#25 \begin{enumerate}

      \item 

      \item 

      \item 

      \item 

    \end{enumerate}

    \item \#31 \begin{enumerate}

        \item 

      \end{enumerate}

    \item \#32 \begin{enumerate}

        \item 


      \end{enumerate}

    \item \#35 \begin{enumerate}



      \end{enumerate}

    \item \#39 \begin{enumerate}



      \end{enumerate}



\end{enumerate}


\end{document}
